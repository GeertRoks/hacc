\documentclass{article}
\usepackage[utf8]{inputenc}
\usepackage[T1]{fontenc}
\usepackage{lmodern}
\usepackage{graphicx}
\graphicspath{ {./imgs/} }
\usepackage{float}
\floatplacement{figure}{H}
\floatplacement{table}{H}
\usepackage{hyperref}
\usepackage[toc,page]{appendix}
\hypersetup{
    colorlinks=true,
    linkcolor=black,
    filecolor=magenta,      
    urlcolor=blue,
    citecolor=blue
}
\setkeys{Gin}{width=\linewidth}
\usepackage[nottoc]{tocbibind}
\usepackage[a4paper,margin=3.5cm]{geometry}
\usepackage[hashEnumerators,smartEllipses,inlineFootnotes,hybrid,definitionLists,pipeTables,tableCaptions,rawHtml=ignore]{markdown} %footnotes


\title{Heterogeneous Accelerated Compute Cluster - ETH Zürich}
\author{Javier Moya, Matthias Gabathuler, Mario Ruiz, Gustavo Alonso}
\date{ETHZ-HACC 2022.1}

\begin{document}

\maketitle

\renewcommand{\abstractname}{Summary}
\begin{abstract}
Under the scope of the AMD University Program, the Heterogeneous Accelerated Compute Clusters (HACCs) is a special initiative to support novel research in adaptive compute acceleration for high-performance computing (HPC). The scope of the program is broad and encompasses systems, architecture, tools, and applications.

HACCs are equipped with the latest Xilinx hardware and software technologies for adaptive compute acceleration research. Each cluster is specially configured to enable some of the world’s foremost academic teams to conduct state-of-the-art HPC research.

Five HACCs have been established at some of world’s most prestigious universities. The first of them was assigned to Prof. Dr. Gustavo Alonso of the Institute for Platform Computing - Systems Group (SG) at the Swiss Federal Institute of Technology Zurich (ETH Zürich) in 2020.
 %\includegraphics[]{basic-training-material/exin-logo.png}
\end{abstract}

\tableofcontents

%\section*{Definition of Scrum}
%\addcontentsline{toc}{section}{Definition of Scrum}
%\markdownInput{the-scrum-guide/definition-of-scrum.md}

%\section{Introduction}

\markdownInput{docs/vocabulary-tex.md}
\markdownInput{docs/hardware-acceleration-platform-tex.md}
%\markdownInput{basic-training-material/other-roles.md}
%\markdownInput{basic-training-material/managing-the-product-backlog.md}
%\markdownInput{basic-training-material/scaling-agile-projects.md}

%\appendix
%\markdownInput{appendix.md}

%\bibliographystyle{unsrt}
%\bibliography{references}
\listoffigures
%\listoftables

\end{document}